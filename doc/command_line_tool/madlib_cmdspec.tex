\documentclass[11pt]{article}
\usepackage{fullpage}
\usepackage{url}
\usepackage{thumbpdf}
\usepackage[ps2pdf=true,colorlinks]{hyperref}
\usepackage[figure,table]{hypcap} % Correct a problem with hyperref
\usepackage{graphicx}
\usepackage{listings}
\usepackage{amsmath}
\hypersetup{
   bookmarksnumbered,
   pdfstartview={FitH},
   citecolor={black},
   linkcolor={black},
   urlcolor={black},
   pdfpagemode={UseOutlines}
}
\newcommand{\versionnumber}{0.2}
\newcounter{mlreqcounter}
\stepcounter{mlreqcounter}


\begin{document}

\lstset{frame=single}
\newenvironment{mlreq}[1]
{%
\newcommand{\mlletter}{#1}
}
{\setcounter{mlreqcounter}{1}}

\newcommand{\mlr}{\textbf{\mlletter\arabic{mlreqcounter}}%
\stepcounter{mlreqcounter}
}

\newenvironment{ttdescription}%
	{%
		\begin{list}{}{\renewcommand{\makelabel}[1]{\texttt{##1}}}%
	}%
	{\end{list}}

\ifx\pdfoutput\undefined % We're not running pdftex
\else
\pdfbookmark{Title Page}{title}
\fi
\newlength{\centeroffset}
\setlength{\centeroffset}{-0.5\oddsidemargin}
\addtolength{\centeroffset}{0.5\evensidemargin}
\thispagestyle{empty}
\vspace*{\stretch{2}}
\noindent\hspace*{\centeroffset}\makebox[0pt][l]
{\begin{minipage}{\textwidth}
\flushright
\Huge\bfseries Specification of\newline
The $\mu$$\alpha$$\delta$lib command-line tool
\noindent
\rule[-1ex]{\textwidth}{5pt}\\[2.5ex]
\end{minipage}
}

{
\vspace{\stretch{1}}
\noindent\hspace*{\centeroffset}\makebox[0pt][l]{\begin{minipage}{\textwidth}
\flushright
\flushright
{\Large\bfseries By Florian Schoppmann (Florian.Schoppmann@emc.com)}\\[2ex]
after discussions with
Aleksander Gorajek (Aleksander.Gorajek@emc.com) \\[.5ex]
based on 
``$\mu$$\alpha$$\delta$lib package manager'' by Gavin Sherry (gavin.sherry@emc.com)
\end{minipage}
}

\vspace{\stretch{1}}
\noindent\makebox[0pt][l]
{\begin{minipage}{\textwidth}
\flushright
Version~\versionnumber{}
\end{minipage}}

\newpage

\tableofcontents
\newpage

\section{Introduction}

Many database systems already support some statistical functions. MADlib is an open source project which provides a common set of functions to several data processing systems, be they Postgres, Greenplum or Hadoop.

An open source project which is hard to install, manage and use is not a compelling one. As such, an important aspect of MADlib will be the ability to easily procure the code and make its functionality available to the data processing system.

This calls for a user-friendly mechanism to aid installation, upgrades and uninstallation.

\ifx\pdfoutput\undefined % We're not running pdftex
\else
\pdfbookmark{Overview}{overview}
\fi
\section{Overview}

The MADlib installation philosophy has two layered components: On the (operating) system level, files and libraries are installed using the platform-native installation infrastructure. Since in general there is a one-to-many correspondence between the operating system and the databases running on it, we need a second tool on that makes the MADlib user-level API available on the database level. This is facilitated by the MADlib command-line tool specified in this document. 

This document is guided by the following important criteria for success.

	\begin{description}
		\item[Don't reinvent the wheel]
			Using the existing and proven tools on each platform ensures that MADlib can be incorporated into existing software-distribution infrastructures, and users immediately feel at home.

		\item[Clean interface] A command line and programmatic interface is provided for making the MADlib user-level API available in a particular database. The command-line tool is easy to use and similar to other tools that perform actions on a single database (e.g., createdb, gp\_dump, gp\_restore, ...).

		\item[Version tracking] It must handle version stamped releases and
		understand the relationship between versions. It must understand
		the difference between major releases and minor releases. That is,
		releases which modify interfaces and those which do not.

		\item[Robustness] The package manager will be used in production
		environments where serious bugs or flaws taint the reputation of the
		project.

		\item[`Just works'] The tool should work intuitively.

		\item[Completeness] The host of minor features which make the tool
		polished and able to be used in unexpected ways.

		\item[Portability] The ability to deploy packages for a variety of data processing platforms. In the initial release, these will be RedHat/CentOS (rpm/yum), Solaris (pkg-get) and Mac OS X (MacPorts) on the system level, and Greenplum and Postgres on the database level. Support will follow soon after for Hadoop and SciDB.

	\end{description}

\section{Package Managers}

There are a significant number of software package managers in the open source
world. Some popular used package managers are the
RPM Package Manager (RPM)\footnote{\url{http://www.rpm.org/}}, 
the Comprehensive Perl Archive Network
(CPAN)\footnote{\url{http://www.cpan.org}},
RubyGems\footnote{\url{http://rubygems.org/}} and
Debian Packages (dpkg)\footnote{\url{http://www.debian.org/distrib/packages}}.

Though they target different users and platforms, these package managers share
several things in common.

\begin{description}
	\item[Well defined interface] A command line tool and API are provided.
	The API allows users to write extensions, such as GUIs.

	\item[Internet-based repository] A centralised location for the publishing
	and retrieval of packages. The client ships with pre-defined methods of
	locating repositories.

	Packages are downloaded via common Internet protocols like HTTP or FTP.

	\item[Versions] Packages can be versioned and version information is
	understood by the package manager.

	\item[Dependencies] A graph of arbitrary complexity can be constructed and
	managed to determine dependencies of one package upon another.

	\item[Robust management] Packages can be downloaded as source or as a
	platform specific binary, they can be built locally, activated, deactivated,
	rebuilt, listed, searched, removed and more.
	
	The installation step itself provides the ability to add scripts for pre and
	post processing.

	The package manager allows users to construct their own packages for local
	use or publishing to the centralised repository.

	\item[Architecture awareness] Packages are availability for specific
	architectures and platforms and the package manager is smart enough to
	install the appropriate package for the host platform.

\end{description}

\ifx\pdfoutput\undefined % We're not running pdftex
\else
\pdfbookmark{Requirements}{requirements}
\fi

\section{Requirements for the (OS-level) MADlib Packages}

\begin{mlreq}{P}
\begin{tabular}{|l|p{133mm}|}
\hline
	\textbf{Requirement} & \textbf{Description} \\
\hline
	\multicolumn{2}{|c|}{\bf Version 1.0} \\
\hline
	\mlr & If possible, the packages will install binaries. Installation from source will only occur where necessary (e.g., MacPorts). \\
\hline
	\mlr & MADlib installation will not interfere with other packages. In particular, uninstalling the databases that MADlib was used in will never leave stray files. \\
\hline

\end{tabular}
\end{mlreq}

\section{Requirements for Command-Line Tool}

\begin{mlreq}{R}
\begin{tabular}{|l|p{133mm}|}
\hline
	\textbf{Requirement} & \textbf{Description} \\
\hline
	\multicolumn{2}{|c|}{\bf Version 1.0} \\
\hline
	\mlr & The command-line tool is able to ``install'' the MADlib user-level API on the database-level. ``Installing'' includes the first setup, up\nobreakdash- and downgrading, and uninstalling. \\
\hline
	\mlr & MADlib in-database functions that depend on code/libraries outside the database (i.e., installed on the operating-system level) perform a version check. Appropriate action (notice, bail out) is taken when a version mismatch between the in-database and OS-level MADlib installation is detected. \\
\hline
	\mlr & Packages may be installed on Postgres and Greenplum Database
		   systems. \\
\hline
	\mlr & The command-line tool comes with basic manual page support.\\

\hline
	\multicolumn{2}{|c|}{\bf After version 1.0} \\
\hline
	\mlr & MADlib can be installed to \texttt{template1} (on Postgres-based databased) so that any
newly created database will automatically contain MADlib. \\
\hline
	\mlr & MADlib will be added to all mainstream package repositories on all supported platforms. \\
\hline
	\mlr & The package management client is able to manage packages for other
		   systems, like Hadoop, SciDB, Oracle and others. \\
\hline

\end{tabular}
\end{mlreq}

\ifx\pdfoutput\undefined % We're not running pdftex
\else
\pdfbookmark{Functional Specification}{spec}
\fi

\section{Functional specification}

\subsection{OS-level Installation}

	Installing MADlib on the OS-level should be no more than (example taken from RedHad/CentOS)
	%
	\begin{lstlisting}
$ yum install madlib
	\end{lstlisting}
	%
	on a single-node setup and
	%
	\begin{lstlisting}
$ gpssh yum install madlib
	\end{lstlisting}
	%
	on a multi-node setup like Greenplum.

\subsection{Database-level Installation: The MADlib command-line tool}


	The basic form for the command-line tool's arguments is:

	\begin{lstlisting}
$ madlib <command> <command-options> <target-uri>
	\end{lstlisting}
	
	Common command options are:
	\begin{ttdescription}
		\item[-s \textit{schemaname}] Use schema \textit{schemaname} when operating on the database. If schemaname is not specified, ``madlib'' will be used by default. 
		\item[-U \textit{username}] Connect to the database as user \textit{username}. Note: This can also be specified as part of the URI.
		\item[-p \textit{password}] Use password \text{password} when connecting to the database. If no password is specified but required to connect, a prompt will be shown on the console.
		\item[-v] Verbose output. If no other arguments are given, report version information.
	\end{ttdescription}

	\subsubsection{Installing MADlib}

		The install command is \texttt{install}. It updates the MADlib in-database API to the MADlib installation installed on the OS-level on disk. Installation can therefore be the first setup, an upgrade or a downgrade.

		\begin{lstlisting}
$ madlib install -s schemaname gpdb://user@mdw:5432
		\end{lstlisting}

	\subsubsection{Uninstalling MADlib}

		The uninstall command is \texttt{uninstall}.

		\begin{lstlisting}
$ madlib uninstall gpdb://user@mdw:5432
		\end{lstlisting}

	\subsubsection{Show version information}

		If no command but the \text{-v} option is given, report version information.

		\begin{lstlisting}
$ madlib -v
		\end{lstlisting}

\ifx\pdfoutput\undefined % We're not running pdftex
\else
\pdfbookmark{Design considerations}{design}
\fi

\section{Design considerations}
	
	\subsection{`Instance', `Database', `Schema'}

	These terms are very charged in the database world. Postgres and Greenplum
	Database have instances, which host databases, which themselves host
	schemas, which contain objects. Oracle on the other hand defines an instance
	as the instantiation of a given database.

	The whole concept breaks down with a system like Hadoop, which does not have
	a formal data schema.

	\subsection{Multi-host environments}

	While Postgres and the Greenplum Database largely present identically to the
	user, the latter is a cluster of networked hosts. To create a C language
	user defined function, the shared library that the C function was compiled
	to needs to be present on all hosts comprising the cluster. This means that
	the installing the package is not just a case of copying the data from the
	package to the file system invoking the package management client.

	For the Greenplum Database, we must assemble a host list and
	call \texttt{gpssh} to run the OS-level package manager on all nodes. While MADlib is not part of mainstream repositories, we must include detailed instructions on how to modify the package-manger configurations on all nodes at once.

	\subsection{User dependencies on packages}

	Users are going to create views and functions which reference objects
	created by MADlib packages. This establishes a dependency outside of the
	MADlib schema. By understanding major and minor version
	changes, that is, those revisions which modify an interface and those which
	do not, the command-line tool can attempt to install a new revision of a
	package without breaking an application which uses it.

\section{Architecture}

\begin{verbatim}
%%%
% Add material of actual substance here
%%%
\end{verbatim}

\ifx\pdfoutput\undefined % We're not running pdftex
\else
\pdfbookmark{Open questions}{questions}
\fi
\section{Open questions}
	
	\subsection{Invocation host vs. installation host(s)}

	Given that we're chiefly interested in network database servers, we find
	ourselves in a networked environment. In this environment, it is normal for
	an administrator to invoke a command from one host which affects another
	host. Should we allow a user to install a package on a host from a host
	which is not part of its database cluster?


\end{document}
