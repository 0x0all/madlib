\documentclass[11pt]{article}
\usepackage{fullpage}
\usepackage{url}
\usepackage{thumbpdf}
\usepackage[ps2pdf=true,colorlinks]{hyperref}
\usepackage[figure,table]{hypcap} % Correct a problem with hyperref
\usepackage{graphicx}
\usepackage{listings}
\hypersetup{
   bookmarksnumbered,
   pdfstartview={FitH},
   citecolor={black},
   linkcolor={black},
   urlcolor={black},
   pdfpagemode={UseOutlines}
}
\newcommand{\versionnumber}{0.1}
\newcounter{mlreqcounter}
\stepcounter{mlreqcounter}


\begin{document}

\lstset{frame=single}
\newenvironment{mlreq}
{%
}
{\setcounter{mlreqcounter}{1}}

\newcommand{\mlr}{\textbf{R\arabic{mlreqcounter}}%
\stepcounter{mlreqcounter}
}

\ifx\pdfoutput\undefined % We're not running pdftex
\else
\pdfbookmark{Title Page}{title}
\fi
\newlength{\centeroffset}
\setlength{\centeroffset}{-0.5\oddsidemargin}
\addtolength{\centeroffset}{0.5\evensidemargin}
\thispagestyle{empty}
\vspace*{\stretch{2}}
\noindent\hspace*{\centeroffset}\makebox[0pt][l]
{\begin{minipage}{\textwidth}
\flushright
\Huge\bfseries Specification of\newline
The $\mu$$\alpha$$\delta$lib package manager
\noindent
\rule[-1ex]{\textwidth}{5pt}\\[2.5ex]
\end{minipage}
}

{
\vspace{\stretch{1}}
\noindent\hspace*{\centeroffset}\makebox[0pt][l]{\begin{minipage}{\textwidth}
\flushright
\flushright
{\Large\bfseries By Gavin Sherry (gavin.sherry@emc.com)}
\end{minipage}
}

\vspace{\stretch{1}}
\noindent\makebox[0pt][l]
{\begin{minipage}{\textwidth}
\flushright
Version~\versionnumber{}.
\end{minipage}}

\newpage

\tableofcontents
\newpage

\section{Introduction}

Many database systems already support some statistical functions. Madlib is an
open source project which provides a comment set of functions to several data
processing systems, be they Postgres, Greenplum or Hadoop.

An open source project which is hard to install, management and use is not a
compelling one. As such, an important aspect of Madlib will be the ability to
easily procure the code and make them available to the data processing system.

This calls for a package management mechanism.

\ifx\pdfoutput\undefined % We're not running pdftex
\else
\pdfbookmark{Overview}{overview}
\fi
\section{Overview}

A package manager hides the administrative complexity of incorporating external
and everchanging software into a production system. The Madlib package manager
has several important criteria for success.

	\begin{description}
		\item[Clean interface] A command line and programmatic interface
		that is easy to use and similar to other package managers.

		\item[Version tracking] It must handle version stamped releases and
		understand the relationship between versions. It must understand
		the difference between major releases and minor releases. That is,
		releases which modify interfaces and those which do not.

		\item[Robustness] The package manager will be used in production
		environments where serious bugs or flaws taint the reputation of the
		project.

		\item[`Just works'] The tool should work intuitively.

		\item[Completeness] The host of minor features which make the tool
		polished and able to be used in unexpected ways.

		\item[Portability] The ability to deploy packages for a variety of
		data processing platforms.i In the initial release, these will be
		Greenplum and Postgres. Support will follow soon after for Hadoop and
		SciDB.

	\end{description}

	A package manager has three significant components: the client tool which
	allows a user to interact with a package of code, the repository of packages
	of code and the local registry of package configuration at a given site.

\section{Related work}

There are a significant number of software package managers in the open source
world. Some popular used package managers are the
RPM Package Manager (RPM)\footnote{\url{http://www.rpm.org/}}, 
the Comprehensive Perl Archive Network
(CPAN)\footnote{\url{http://www.cpan.org}},
RubyGems\footnote{\url{http://rubygems.org/}} and
Debian Packages (dpkg)\footnote{\url{http://www.debian.org/distrib/packages}}.

Though they target different users and platforms, these package managers share
several things in common.

\begin{description}
	\item[Well defined interface] A commmand line tool and API are provided.
	The API allows users to write extensions, such as GUIs.

	\item[Internet-based repository] A centralised location for the publishing
	and retrieval of packages. The client ships with pre-defined methods of
	locating repositories.

	Packages are downloaded via common Internet protocols like HTTP or FTP.

	\item[Versions] Packages can be versioned and version information is
	understood by the package manager.

	\item[Dependencies] A graph of abitrary complexity can be constructed and
	managed to determine dependencies of one package upon another.

	\item[Robust management] Packages can be downloaded as source or as a
	platform specific binary, they can be built locally, activated, deactivated,
	rebuilt, listed, searched, removed and more.
	
	The installation step itself provides the ability to add scripts for pre and
	post processing.

	The package manager allows users to construct their own packages for local
	use or publishing to the centralised repository.

	\item[Architecture awareness] Packages are availability for specific
	architectures and platforms and the package manager is smart enough to
	install the appropriate package for the host platform.

\end{description}

\ifx\pdfoutput\undefined % We're not running pdftex
\else
\pdfbookmark{Requirements}{requirements}
\fi

\section{Requirements}

\begin{mlreq}
\begin{tabular}{|l|p{133mm}|}
\hline
	\textbf{Requirement} & \textbf{Description} \\
\hline
	\multicolumn{2}{|c|}{\bf Version 1.0} \\
\hline
	\mlr & Each method in Madlib is a seperate package. \\
\hline
	\mlr & All packages are versioned. \\
\hline
	\mlr & A command line package management client is able to install
	       and uninstall packages. \\
\hline
	\mlr & A registry of package status is maintained on hosts which run
		   package management client. \\
\hline
	\mlr & Packages may be installed on Postgres and Greenplum Database
		   systems. \\
\hline
	\mlr & The command line package management client can list installed
		   packages and their version. \\

\hline
	\mlr & The command line package management client is able to perform
		   minor version upgrades of packages. \\

\hline
	\mlr & The package management client is able to ensure that, when the target
		   is a clustered database like the Greenplum Database, it can
		   install the package on all nodes in a cluster. \\
\hline
	\mlr & The package management client comes with basic manual page support.\\

\hline
	\multicolumn{2}{|c|}{\bf After version 1.0} \\
\hline
	\mlr & An application programming interface is provided. It exposes all
		   functionality available in the command line tool. \\
\hline
	\mlr & The package management client is able to download packages from an
		   online repository. \\
\hline
	\mlr & The package management client is able to manage packages for other
		   systems, like Hadoop, SciDB, Oracle and others. \\
\hline
	\mlr & The package management client is able to manage its repository across
		   hosts when it runs in an environment which has multiple master nodes,
		   such as that of the Greenplum Database or replicated Postgres
		   systems. \\
\hline
	\mlr & When installing a package on a clustered database system, the
		   distributed installation is fault tolerant. \\

\hline
	\mlr & A detailed tutorial and API document is provided. \\
\hline

\end{tabular}
\end{mlreq}

\ifx\pdfoutput\undefined % We're not running pdftex
\else
\pdfbookmark{Functional Specification}{spec}
\fi

\section{Functional specification}

	The basic form for the package management client's arguments is:

	\begin{lstlisting}
$ mlpm <command> <command-options> <target-uri>
	\end{lstlisting}

	\subsection{Installing a package}

		The install command is \texttt{-i | --install}.

		\begin{lstlisting}
$ mlpm -i sketches-1.0-1.x86.mlpm gpdb://mdw:5432
		\end{lstlisting}

	\subsection{Uninstalling a package}

		The uninstall command is \texttt{-e | --erase}.

		\begin{lstlisting}
$ mlpm -e sketches-1.0-1 gpdb://mdw:5432
		\end{lstlisting}

	\subsection{Activating a package}

		Installing a package installs the package on the file system, but it
		doesn't setup the package in a database. To do this, the package must be
		activated.

		The activate command is \texttt{-a | --activate}

		\begin{lstlisting}
$ mlpm -a sketches-1.0-1 -s analytics gpdb://mdw:5432/production
		\end{lstlisting}

		Notice here that the URI specifying the target has a database,
		`production', specified. The \texttt{-s | --schema} option specifies the
		schema where the database objects should be created.

	\subsection{Deactivating a package}

		The deactivate command is \texttt{-d | --deactivate}

		\begin{lstlisting}
$ mlpm -d sketches-1.0-1 gpdb://mdw:5432/production
		\end{lstlisting}

	\subsection{Upgrade}

		The upgrade command allows users to upgrade a package in place. The
		command is \texttt{-u | --upgrade}.

		\begin{lstlisting}
$ mlpm -u sketches-1.0-2.x86.mlpm gpdb://mdw:5432
		\end{lstlisting}

		This command is functionally equivalent to uninstalling the old package
		and installing and activating the new one.

	\subsection{List all installed packages}

		The list command is \texttt{-l | --list}.

		\begin{lstlisting}
$ mlpm -l gpdb://mdw:5432
		\end{lstlisting}

\ifx\pdfoutput\undefined % We're not running pdftex
\else
\pdfbookmark{Design considerations}{design}
\fi

\section{Design considerations}
	
	\subsection{`Instance', `Database', `Schema'}

	These terms are very charged in the database world. Postgres and Greenplum
	Database have instances, which host databases, which themselves host
	schemas, which contain objects. Oracle on the other hand defines an instance
	as the instantiation of a given database.

	The whole concept breaks down with a system like Hadoop, which does not have
	a formal data schema.

	\subsection{Multi-host environments}

	While Postgres and the Greenplum Database largely present identically to the
	user, the latter is a cluster of networked hosts. To create a C language
	user defined function, the shared library that the C function was compiled
	to needs to be present on all hosts comprising the cluster. This means that
	the installing the package is not just a case of copying the data from the
	package to the file system invoking the package management client.

	It seems reasonable that we need to define a copy step per target database
	system. That is, for Postgres, the UNIX \texttt{cp(1)} command is
	sufficient; for the Greenplum Database, we must assemble a host list and
	call \texttt{gpscp}.

	\subsection{User dependencies on packages}

	Users are going to create views and functions which reference objects
	created by Madlib packages. This estalishes a dependency outside of the
	Madlib package manager's control. By understanding major and minor version
	changes, that is, those revisions which modify an interface and those which
	do not, the package manager can attempt to install a new revision of a
	package without breaking an application which uses it.

\section{Architecture}

\begin{verbatim}
%%%
% Add material of actual substance here
%%%
\end{verbatim}

\ifx\pdfoutput\undefined % We're not running pdftex
\else
\pdfbookmark{Open questions}{questions}
\fi
\section{Open questions}
	
	\subsection{Invocation host vs. installation host(s)}

	Given that we're chiefly interested in network database servers, we find
	ourselves in a networked environment. In this environment, it is normal for
	an administrator to invoke a command from one host which affects another
	host. Should we allow a user to install a package on a host from a host
	which is not part of its database cluster?

	\subsection{The package registry}

	The previous issue implies another issue: in any distributed environment, we
	need to be concerned with representation of state across the network.

	When a package in installed, the status of that package is recorded in the
	package management registry.

	\subsection{Wrap or implement from scratch}

	While not nearly as complex as a package manager like RPM, the Madlib
	package manager will not be trivial to implement. The question, then, is: do
	we use an existing package manager internally and present a simplified
	interface or do we write a new package manager from scratch?


\end{document}
